% Options for packages loaded elsewhere
\PassOptionsToPackage{unicode}{hyperref}
\PassOptionsToPackage{hyphens}{url}
%
\documentclass[
]{article}
\title{Project 1 calculations}
\author{}
\date{\vspace{-2.5em}}

\usepackage{amsmath,amssymb}
\usepackage{lmodern}
\usepackage{iftex}
\ifPDFTeX
  \usepackage[T1]{fontenc}
  \usepackage[utf8]{inputenc}
  \usepackage{textcomp} % provide euro and other symbols
\else % if luatex or xetex
  \usepackage{unicode-math}
  \defaultfontfeatures{Scale=MatchLowercase}
  \defaultfontfeatures[\rmfamily]{Ligatures=TeX,Scale=1}
\fi
% Use upquote if available, for straight quotes in verbatim environments
\IfFileExists{upquote.sty}{\usepackage{upquote}}{}
\IfFileExists{microtype.sty}{% use microtype if available
  \usepackage[]{microtype}
  \UseMicrotypeSet[protrusion]{basicmath} % disable protrusion for tt fonts
}{}
\makeatletter
\@ifundefined{KOMAClassName}{% if non-KOMA class
  \IfFileExists{parskip.sty}{%
    \usepackage{parskip}
  }{% else
    \setlength{\parindent}{0pt}
    \setlength{\parskip}{6pt plus 2pt minus 1pt}}
}{% if KOMA class
  \KOMAoptions{parskip=half}}
\makeatother
\usepackage{xcolor}
\IfFileExists{xurl.sty}{\usepackage{xurl}}{} % add URL line breaks if available
\IfFileExists{bookmark.sty}{\usepackage{bookmark}}{\usepackage{hyperref}}
\hypersetup{
  pdftitle={Project 1 calculations},
  hidelinks,
  pdfcreator={LaTeX via pandoc}}
\urlstyle{same} % disable monospaced font for URLs
\usepackage[margin=1in]{geometry}
\usepackage{graphicx}
\makeatletter
\def\maxwidth{\ifdim\Gin@nat@width>\linewidth\linewidth\else\Gin@nat@width\fi}
\def\maxheight{\ifdim\Gin@nat@height>\textheight\textheight\else\Gin@nat@height\fi}
\makeatother
% Scale images if necessary, so that they will not overflow the page
% margins by default, and it is still possible to overwrite the defaults
% using explicit options in \includegraphics[width, height, ...]{}
\setkeys{Gin}{width=\maxwidth,height=\maxheight,keepaspectratio}
% Set default figure placement to htbp
\makeatletter
\def\fps@figure{htbp}
\makeatother
\setlength{\emergencystretch}{3em} % prevent overfull lines
\providecommand{\tightlist}{%
  \setlength{\itemsep}{0pt}\setlength{\parskip}{0pt}}
\setcounter{secnumdepth}{-\maxdimen} % remove section numbering
\ifLuaTeX
  \usepackage{selnolig}  % disable illegal ligatures
\fi

\begin{document}
\maketitle

\hypertarget{exponentialfuxf6rdelningen}{%
\section{Exponentialfördelningen}\label{exponentialfuxf6rdelningen}}

\hypertarget{tuxe4thetsfunktionen}{%
\subsection{Täthetsfunktionen}\label{tuxe4thetsfunktionen}}

\[
p_X(x) = \lambda e^{-\lambda x}, \quad x \ge 0
\]

\hypertarget{likelihood-fuxf6r-en-observation}{%
\subsection{Likelihood för en
observation}\label{likelihood-fuxf6r-en-observation}}

\[
L(\lambda) = \lambda e^{-\lambda x}
\]

\hypertarget{log-likelihood-fuxf6r-en-observation}{%
\subsection{Log-likelihood för en
observation}\label{log-likelihood-fuxf6r-en-observation}}

\[
\begin{aligned}
l(\lambda) &= \ln(\lambda e^{-\lambda x}) \\
           &= \ln(\lambda) + \ln(e^{-\lambda x}) \\
           &= \ln(\lambda) - \lambda x \ln(e) \\
           &= \ln(\lambda) - \lambda x 
\end{aligned}
\]

\hypertarget{log-likelihood-fuxf6r-hela-urvalet}{%
\subsection{Log-likelihood för hela
urvalet}\label{log-likelihood-fuxf6r-hela-urvalet}}

\[
\begin{aligned}
l_n(\lambda) &= \sum_{i=1}^{n}(\ln(\lambda) - \lambda x) \\
             &= n\ln(\lambda) - \lambda\sum_{i=1}^{n}x_i
\end{aligned}
\]

\hypertarget{fuxf6rsta-derivatan-av-likelihood-funktionen-fuxf6r-urvalet}{%
\subsection{Första derivatan av likelihood funktionen för
urvalet}\label{fuxf6rsta-derivatan-av-likelihood-funktionen-fuxf6r-urvalet}}

\[
\begin{aligned}
l'_n(\lambda) &= \frac{d }{d\lambda}(n\ln(\lambda) - \lambda\sum_{i=1}^{n}x_i) \\
              &= n\lambda^{-1} - \sum_{i=1}^{n}x_i 
\end{aligned}
\]

\hypertarget{ml-skattningen-ges-utav-att-luxf6sa}{%
\subsection{ML skattningen ges utav att
lösa}\label{ml-skattningen-ges-utav-att-luxf6sa}}

\[
\begin{aligned}
l'_n(\lambda) &= n\lambda^{-1} - \sum_{i=1}^{n}x_i = 0 \\
              &\Leftrightarrow \sum_{i=1}^{n}x_i = \frac{n}{\lambda} \\
              &\Leftrightarrow \hat\lambda = \frac{n}{\sum_{i=1}^{n}x_i}
\end{aligned}
\]

\hypertarget{fuxf6r-att-beruxe4kna-fisherinformationen-behuxf6ver-vi-andraderivatan-utav-log-likelihood-funktionen}{%
\subsection{För att beräkna fisherinformationen behöver vi
andraderivatan utav log-likelihood
funktionen}\label{fuxf6r-att-beruxe4kna-fisherinformationen-behuxf6ver-vi-andraderivatan-utav-log-likelihood-funktionen}}

\[
\begin{aligned}
l''_n(\lambda) &= \frac{d }{d\lambda}(n\lambda^{-1} - \sum_{i=1}^{n}x_i) \\
               &= -n\lambda^{-2} 
\end{aligned}
\]

\hypertarget{fisherinformationen-fuxf6r-urvalet-blir-duxe5}{%
\subsection{Fisherinformationen för urvalet blir
då}\label{fisherinformationen-fuxf6r-urvalet-blir-duxe5}}

\[
\begin{aligned}
I_n(\lambda) &= - E[l''_n(\lambda)] \\
             &= - E[-n\lambda^{-2}] \\
             &= n\lambda^{-2}
\end{aligned}
\]

\hypertarget{medelfelet-fuxf6r-ml-skattningen-ges-duxe5-utav}{%
\subsection{Medelfelet för ML-skattningen ges då
utav}\label{medelfelet-fuxf6r-ml-skattningen-ges-duxe5-utav}}

\[
\begin{aligned}
Sd(\hat\lambda) &= I_n(\hat\lambda)^{-1/2} \\
                &= (n\hat\lambda^{-2})^{-1/2} \\
                &= \hat\lambda n^{-1/2}
\end{aligned}
\]

\hypertarget{binomialfuxf6rdelningen}{%
\section{Binomialfördelningen}\label{binomialfuxf6rdelningen}}

\hypertarget{tuxe4thetsfunktionen-1}{%
\subsection{Täthetsfunktionen}\label{tuxe4thetsfunktionen-1}}

\[
p_X(k) = {n \choose k} p^k(1 - p)^{n - k}, \quad k=\text{Antal utfall}
\]

\hypertarget{likelihood-fuxf6r-hela-urvalet}{%
\subsection{Likelihood för hela
urvalet}\label{likelihood-fuxf6r-hela-urvalet}}

\[
L_n(p) = {n \choose k} p^k(1 - p)^{n - k}
\]

\hypertarget{log-likelihood-fuxf6r-hela-urvalet-1}{%
\subsection{Log-likelihood för hela
urvalet}\label{log-likelihood-fuxf6r-hela-urvalet-1}}

\$\$

\begin{aligned}
l_n(p) &= \ln({n \choose k} p^k(1 - p)^{n - k}) \\
     &= \ln({n \choose k}) + \ln(p^k) + \ln((1 - p)^{n - k}) \\
     &= \ln({n \choose k}) + k\ln(p) + (n - k)\ln(1-p)
           
\end{aligned}

\$\$

\hypertarget{fuxf6rsta-derivatan-utav-log-likelihood-funktionen-fuxf6r-hela-urvalet}{%
\subsection{Första derivatan utav log-likelihood funktionen för hela
urvalet}\label{fuxf6rsta-derivatan-utav-log-likelihood-funktionen-fuxf6r-hela-urvalet}}

\[
\begin{aligned}
l'_n(p) &= \frac{d }{dp}(\ln({n \choose k}) + k\ln(p) + (n - k)\ln(1-p)) \\
            &= kp^{-1} - (n - k)(1 - p)^{-1}
\end{aligned}
\]

\hypertarget{ml-skattninges-ges-utav-att-luxf6sa-ekvationen-fuxf6r-p}{%
\subsection{ML-skattninges ges utav att lösa ekvationen för
p}\label{ml-skattninges-ges-utav-att-luxf6sa-ekvationen-fuxf6r-p}}

\[
\begin{aligned}
l'_n(p) &= kp^{-1} - (n - k)(1 - p)^{-1} = 0 \\
        &\Leftrightarrow kp^{-1} = (n - k)(1 - p)^{-1} \\
        &\Leftrightarrow (1-p)k = p(n-k) \\
        &\Leftrightarrow k - pk = pn - pk \\
        &\Leftrightarrow k = pn \\
        &\Leftrightarrow \hat p = \frac{k}{n}
\end{aligned}
\] \#\# För att beräkna fisherinformationen behöver vi andraderivatan
utav log-likelihood funktionen

\[
\begin{aligned}
l''_n(p) &= \frac{d }{dp}(kp^{-1} - (n - k)(1 - p)^{-1}) \\
         &= -kp^{-2} - (n - k)(1 - p)^{-2}
\end{aligned}
\]

\hypertarget{fisherinformationen-fuxf6r-hela-urvalet-blir-duxe5}{%
\subsection{Fisherinformationen för hela urvalet blir
då}\label{fisherinformationen-fuxf6r-hela-urvalet-blir-duxe5}}

\[
\begin{aligned}
I_n(p) &= - E[l''_n(p)] \\
             &= -E[-kp^{-2} - (n - k)(1 - p)^{-2}] \\
             &= -E[\frac{-k + 2kp - np^2}{p^2(1-p)^2}] \\
\end{aligned}
\]

Eftersom \(E[k] = np\) får vi

\[
\begin{aligned}
I_n(p) &= - E[l''_n(p)] \\
             &= -E[-kp^{-2} - (n - k)(1 - p)^{-2}] \\
             &= -E[\frac{-k + 2kp - np^2}{p^2(1-p)^2}] \\
             &= \frac{np + np^2}{p^2(1-p)^2} \\
             &= \frac{n}{p(1-p)}
\end{aligned}
\]

\hypertarget{medelfelet-fuxf6r-ml-skattningen-ges-duxe5-utav-1}{%
\subsection{Medelfelet för ML-skattningen ges då
utav}\label{medelfelet-fuxf6r-ml-skattningen-ges-duxe5-utav-1}}

\[
\begin{aligned}
Sd(\hat p) &= I_n(\hat p)^{-1/2} \\
                &= \sqrt{\frac{\hat p(1-\hat p)}{n}} \\
\end{aligned}
\]

\end{document}
